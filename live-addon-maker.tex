%\documentclass{lni}
% in englisch stattdessen:
\documentclass[english]{lni}

\IfFileExists{latin1.sty}{\usepackage{latin1}}{\usepackage{isolatin1}}

\usepackage{graphicx}
\usepackage{url}
\usepackage{fancyhdr}
\usepackage{listings} %if lstlistings is used
\usepackage{changepage} %for changing topmargin on first page
\usepackage[figurename=Abb., tablename=Tab., small]{caption}[2008/04/01]
\renewcommand{\lstlistingname}{List.}    % Listingname heißt nun List. 

%Beginn der Seitenzählung für diesen Beitrag
\setcounter{page}{11}

%Kopfzeileneinstellungen
\pagestyle{fancy}
\fancyhead{} % Löscht alle Kopfzeileneinstellungen
\fancyhead[RO]{\small Nicco Kunzmann, \linebreak Lecture Notes in Informatics (LNI), Gesellschaft für Informatik, Bonn $<$2017$>$ \hspace{5pt} \thepage \hspace{0.05cm}}
\fancyfoot{} % Löscht alle Fußzeileneinstellungen
\renewcommand{\headrulewidth}{0.4pt} %Linie unter Kopfzeile 
\setcounter{footnote}{0}

\author{Nicco Kunzmann\footnote{Hasso-Plattner-Institut, Universität Potsdam, Heinrich-von-Kleist-Str. 24, 14482 Potsdam, Nicco.Kunzmann@student.hpi.de} 
%\, Vorname2 Nachname2\footnote{Einrichtung/Universität, Abteilung, Anschrift, Postleitzahl Ort, emailadresse@author2} \ und weitere Autorinnen und Autoren in der gleichen Notation
}
\title{Fine-Grained Customization of Ubuntu Live Systems}
\begin{document}
\maketitle
%Überschrift des Literaturverzeichnisses - delete this line in english
\renewcommand{\refname}{Literaturverzeichnis}
\setcounter{footnote}{2} %Auf Anzahl der AutorInnen setzen, damit die weitere Nummerierung der Fußnoten passt

\begin{abstract}
Existing customization tools for Ubuntu live images focus on changing the whole file system.
In this paper, we discuss the findings from the Live Addon Maker:
It uses less space and time to customize ISO images.
Resulting customizations are independent, smaller and can be combined.
This lowers the entry barrier for live system customization.
\end{abstract}
\begin{keywords}
\end{keywords}

\section{Motivation}

Students in programming clubs like CoderDojos\footnote{\url{https://coderdojo.com}} use special software to write computer programs.
To keep the entrance barrier low, these clubs do not charge for attendance.
As a result, the computers use Linux, preferably on USB-devices.
An installation of Ubuntu on a USB-device boots slowly or not, see Section~\ref{sec:related}.
Therefore, we created a tool to create files which extend the live system on the  USB-device, so called addons.
They can be installed by copying them onto the stick.
They may contain software, files and configuration.

\section{Related Work}
\label{sec:related}

There are several ways to customize a Ubuntu on a USB-device.
In the following, we discuss the installation on the USB-device, creating custom ISO images, extending the squashfs.filesystem, and using the persistence of the live system.

\subsection{The Released Live Images}

We first used the Live images for the programming event for Children "CoderDojo" at the 33C3\footnote{\url{https://events.ccc.de/congress/2016/wiki/Projects:Junghackertag}}.
They booted in below five minutes.
However, additional software needed to be installed on every system by hand, every time the system  was rebooted.
This is in contrast to the idea of having a working system on a USB-drive, that is fully fuctional for our purposes once it is booted.

\subsection{Installation of Ubuntu on a USB device}

Predefined Ubuntu images can be installed on a USB device.
We performed such an installation on a USB 2.0 device with 5MB/s writing and 10mb/s reading speed.
Lubuntu 16.4\footnote{\url{http://cdimage.ubuntu.com/lubuntu/releases/xenial/}} did not boot to desktop. 
Ubuntu Mate 16.4\footnote{\url{https://ubuntu-mate.org}} took 5 minutes to boot.
Then, it was unresponsive with every click we made, regardless of the specific action.

We concluded, that an installation on these USB devices was too slow and frustrating to use.

\subsection{Creating Custom ISO Images}
\label{sec:custom}

Since CoderDojo Potsdam\footnote{\url{http://coderdojopotsdam.github.io/}} is not the only Dojo using USB devices to boot a custom image from, we decided to create a software which should make it possible to extend live-images to any needs. 
Since everyone might have different requirements for the installed software on the system, we allow users to choos the software.
We created CodersOS\footnote{\url{https://github.com/CodersOS/}} for the following work flow:

\begin{enumerate}
  \item Choose the software from a website\footnote{\url{https://codersos.github.io/create}}
  \item Build the image\footnote{\url{https://github.com/CodersOS/image-creator-server}}
  \item Download and flash the ISO file to the USB device.
\end{enumerate}

Despite implementing the process fully, some ISO images did not boot because the 
squashed file system could not be mounted.
It can be assumed that some Ubuntu flavors can boot this way but looking at Ubuntu Remix, Xubuntu and Meilix, we did not find a reliable way to boot them.

\subsection{Live Image Persistence}
\label{sec:pers}

When working on a live system, saving the work that was done or persisting changes that were made through the installation of new software, is complicated or impossible.
When an Ubuntu live ISO boots, one can pass parameters such as \texttt{PERSISTENCE}.
If this parameter is passed, the live image looks for the file \texttt{casper-rw} with an ext2 file system in it.
This file system is mounted in read-write mode\footnote{To find this behavior, unzip the file \texttt{casper/initrd.lz} and have a look at the casper shell script or preferably \url{https://help.ubuntu.com/community/LiveCD/Persistence}} on an AUFS layer above the filesystem.squashfs.
Any change of the file system is saved to this file.

When we tested this method we found out, that the system ``Ubuntu 16.10 Destop amd64" did not boot after installing new software and saving changes in the persistence file.
This approach proved to be not as reliable as we hoped.

\subsection{Ubuntu Customization Kit}

There is a variety of customization programs for Ubuntu distributions.
We had a look at the Ubuntu Customization Kit "UCK"\footnote{\url{http://uck.sourceforge.net/}}.
The project is discontinued, so we decided against using this as a basis for the  image customization.


\subsection{Customizer}
\label{sec:customizer}

Customizer\footnote{\url{https://github.com/kamilion/customizer}} is a software to change existing ISO images.
This software features a command line interface (CLI) and a graphical user interface (GUI).
Both can be used.
We wanted to create an automated customization over a website, see Section~\cite{sec:custom}.
We could not understand the CLI at this time.
Thus, we discarded the tool.
Furthermore, this tool uses the change-root environment, which is not available in docker.
Having gathered the knowledge of Section~\ref{sec:live}, we can understand Customizer and compare it with Live Addons.

\section{Creating Live System Addons}
\label{sec:live}


\subsection{The Boot Process}
\label{sec:boot}
Live systems such as Ubuntu 16.10 amd64, use casper scripts for booting.
This describes the boot process in shortened form:
\begin{enumerate}
  \item The kernel is loaded.
        It loads the file \texttt{casper/initrd.lz}.
        This is a compressed minimal Linux file structure with shell scripts.
  \item The \texttt{init} script is executed.
        It passes the boot parameters further to the \texttt{casper} script.
  \item The \texttt{casper} script looks for all mountable files with a \texttt{.ext2}, \texttt{.squashfs} and folders with a \texttt{.dir} extension in the \texttt{casper} folder.
        These are mounted in order of the extension and the name.
  \item These mounted folders are used as a basis for Another Union File (AUFS) in read-only mode.
        When a program wants to open a file in AUFS, the file is looked for in the given order in the mounted file systems.
        The first occurrence is used.
  \item In case persistence is enables, see Section~\ref{sec:pers}, the \texttt{casper-rw} file is included in the AUFS mount as first writable file system.
  \item A script changes the root of the file system into the AUFS folder.
        Then, \texttt{/sbin/init} is executed and the system boots.
\end{enumerate}

The live system is stored in the \texttt{filesystem.squashfs} file.
As we can see, the boot process does handle a set of these files simultaneously.
Thus, we if we create another \texttt{.squashfs} file, it is mounted and the files are accessible when the user logs in.

\section{Implementation}

Adding a program or files to the live image before boot can be done with a simple copy and paste of a file system file or directory into the \texttt{casper} folder.
Section~\ref{sec:boot} talks about how the file is integrated into the live system.

In order to create such files, we created a command line tool named ``Live Addon Maker".
This tool creates the the addon files.
In Section~\ref{sec:requirements}, we talk about the interface design decisions.
In Section~\ref{sec:implementation} we talk about implementation details and challenges.

\subsection{Requirements}
\label{sec:requirements}

We isolated three different stages at which users may want to change the live system.
These stages are reflected when creating an addon.

\begin{itemize}
  \item Startup:
        When the system boots, a command is executed.
        For example, this command can delete files or connect to wifi.
  \item Persistent:
        These commands and files are persisted in the addon file system.
        They appear as read-only in the live system.
        This could be an installed package or information for the user.
  \item Volatile:
        These keep the addon small and reduce conflict between addons.
        These can be files used by the persistent commands to install something.
        These files are obsolete, once the installation took place.
        An example are source code files or temporary files.
\end{itemize}

When creating one addon, we can use all these stages.
E.g. we can update the package list and not store this information in the addon (volatile). Then, we can install a package (persistent) and execute a script on start-up which uses the installed package.

When booting the system, we also recognized, that a maximum number of addons lower than 10 is supported by the system.
This is due to a shortage of loop devices which are required to mount \texttt{squashfs} files.
To support more addons, we created a script to merge several addons into one.

Also, user may want to deliver the live iso image with the addon built in.
Thus, there should be a way to add addons to existing ISO images.

In the following Section~\ref{sec:implementation}, we discuss how we implemented these requirements.

\subsection{Interface and Implementation}
\label{sec:implementation}

The interface is a direct conclusion from Section~\ref{sec:requirements}.
We can add files in persistent and volatile mode, execute commands in persistent, volatile mode and on start-up.
In this section, we discuss how we implemented the desired behavior and discuss our findings.

When an addon is created for an ISO image, we need to extract the file system from this image.
It is possible to unpack the image and using several gigabytes of space and minutes of time.
So, we decided to mount the image and the \texttt{filesystem.squashfs} file in the iso.
This takes almost no time and only a few kilobytes of space.

Based on the mounted file system, we can execute commands in a change-root environment.
These commands can be volatile and persistent.
However, commands shall modify the underlying system, e.g. add packages or change settings.
A solution to this was already provided by the \texttt{casper} script.
The different \texttt{squashfs} files are layered by AUFS.
We added a writable layer on top of the \texttt{filesystem.squashfs}.
All changes by commands are stored there.

When an addon is created, we select the layers of the AUFS which shall be persistent.
These layers are squashed into an addon with the squashfs-tools.

In the following section, we compare this approach to other customization approaches.

\section{Evaluation and Future Work}

In this section, we want to compare the Live Addon Maker to existing tools and point out and evaluate the differences.
We also propose a set of features which we think, can be evaluated in the future in combination with existing software.

In Section~\ref{sec:customizer}, we describe a tool which has similar functionality as the Live Addon Maker.
While the live-addon maker creates files which can be added to the live system,
the Customizer modifies the existing file system.
This requires extracting the file system and uses more space.
For the common use-cases, like adding packages, the Customizer has a set of parameters and GUI options available.
This is not the case with the live-addon-maker.
In future, while the use-case seems to be slightly different, the tools can combined to allow addons and file system modification.
In order to save space in the future, the Customizer could be modified to mount the existing file system instead of extracting the content.

In Section~\ref{sec:custom}, we describe the ability to create custom build iso images via a website.
This is the lowest entry barrier, as more people are capable of using a website and following a tutorial boot from a downloaded ISO than using the command line interface.
However, using the change-root environment, it is not possible to use docker as a build environment.
As such, both implementations are incompatible.
However, if we wanted to combine both approaches, we can speed up image customization from a website by storing independent modifications as addons.
It is much faster to create an addon once and add it to the iso later than executing the commands to modify this addon.
Furthermore, the ability to switch between persistent and nonpersistent mode allows for smaller file systems.

Independent addons can be combined.
This makes it possible to customize the images by copy and pasting a set of addons.
This is faster and more accessible than installing a customization software.

We can see that the Live Addon Maker contributes new approaches and views to the  tool set of customizing Ubuntu live systems.
Implemented features can be a basis for existing tools to open up to new use-cases and become more efficient.
Until then, this software solves the need for at least one Programming Club.

\section{Conclusion}

In this paper, we presented a tool to create addons for Ubuntu live systems.
While existing tools focus on modifying the whole file system, the Live Addon Maker creates small files in addition to the existing system.

Using the AUFS layers makes the addon creation fast and space efficient.
This also allows to select more fine grained which modifications are required in the resulting image.
These insights can be used to improve existing live system customization tools.

\bibliographystyle{lnig}
\bibliography{lniguide}

\end{document}



